\documentclass[../main.tex]{subfiles}
\begin{document}
\chapter{Problema de investigación}
\section{Planteamiento del problema}
Se llevará a cabo un estudio sobre el efecto de un campo magnético externo, incluyendo el análisis Hamiltoniano de materia en un sistema cerrado, la producción de oscilaciones gigante-Rabi mediante el análisis del Hamiltoniano de interacción y de materia, la dinámica del sistema teniendo en cuenta canales de disipación, la emisión bundle de $N$-fonones y el espectro de emisión fonónica. Este estudio se realizará en una cavidad monomodal QED acústica con un punto cuántico, considerando la base excitónica más completa que incluye excitones brillantes y oscuros.

\section{Objetivo general}
Estudiar el efecto de un campo magnético externo en la producción de oscilaciones gigante-Rabi.
\section{Objetivos específicos}
\begin{enumerate}[(a)]
	\item Caracterizar la din\'amica cu\'antica cuando se incluye la interacción del campo magnético.
	\item Estudiar el rol que juega el campo magnético en la producción de oscilaciones gigante-Rabi.
	\item Explorar si al aplicar un campo magnético externo se generan bundles de $N$-fonones en espectro de emisi\'on.
\end{enumerate}

\section{Metodología}
El presente proyecto propone un estudio te\'orico que considera tres etapas para alcanzar los
objetivos propuestos:
\begin{enumerate}
	\item El efecto del campo magnético en un sistema cerrado de materia, utilizando un análisis Hamiltoniano para describir y entender las dinámicas del sistema. La producción de oscilaciones gigante-Rabi, que son oscilaciones coherentes entre estados de un sistema cuántico, inducidas por la interacción con un campo electromagnético. Se realizará un análisis del Hamiltoniano de interacción y de materia para entender este fenómeno.
	\item La dinámica del sistema considerando al menos cuatro canales de disipación. Los canales de disipación son formas en las que la energía puede salir del sistema, y su estudio es crucial para entender la evolución temporal del sistema.
	\item La emisión de un conjunto de $N$-fonones. ya que su emisión puede proporcionar información valiosa sobre las propiedades del sistema. Por medio del espectro de emisión fonónica, que es la distribución de energía de los fonones emitidos por el sistema.
\end{enumerate}

Todo esto se llevará a cabo en el contexto de una cavidad monomodal de electrodinámica cuántica (QED) acústica que contiene un punto cuántico. En este sistema, se considerará la base excitónica más rica, incluyendo tanto excitones brillantes como oscuros.
%
\end{document}