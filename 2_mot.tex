\documentclass[main.tex]{subfiles}
\begin{document}
\chapter{Motivación}
Sistemas microcavidad, o cavidad, conteniendo puntos cuánticos, en inglés quantum dots (QDs), han sido explorados con intensidad en décadas recientes \parencite{Reitzenstein2012}. Dentro de estos sistemas se han estudiado efectos importantes como el control de las tasas de emisión espontánea \parencite{Bayer2001}, la obtención de un régimen de acoplamiento fuerte \parencite{Reithmaier2004}, la producción de fuentes de fotones individuales \parencite{Michler2000}, en general, con el fin de tener control de las propiedades cuánticas \parencite{Jimenez2017}.

Dichos sistemas cavidad-QD son importantes porque se pueden usar como candidatos, por mencionar algunos, prometedores para: la manipulación controlada de los procesos de interacción radiación-materia y el desarrollo de dispositivos que pueden ser usados en computación cuántica \parencite{Jimenez2017}. Además, una de las posibilidades principales es que se puede controlar la condición de resonancia entre el modo de la cavidad y los estados del QD, esto se puede lograr variando la temperatura de la red de QDs \parencite{Reithmaier2004}, o usando un campo eléctrico  $\vec{\mathbf{E}}$ junto con el efecto Stark cuántico confinado o usando  un campo magnético $\vec{\mathbf{B}}$ junto con el desplazamiento diamagnético \parencite{Jimenez2017}.

Sistemas comunes de estudio consideran únicamente estados excitón\footnote{Cuasipartícula formada por interacción Coulombiana entre un electrón en la banda de conducción y el hueco que dejó en la banda de valencia. Es posible formar excitones si la energía suministrada al electrón, que está inicialmente en la banda de valencia, no es tan grande como para volverlo libre.} que interactúan con luz, también conocidos como excitones brillantes. Estos estados presentan características, como por ejemplo, pueden decaer emitiendo un fotón y su interacción con un campo electromagnético hace que su tiempo de vida sea muy corto, ideal para comprobación de información cuántica \parencite{Poem2010}. 

%\subsubsection{Sistemas QD semiconductores}
En QDs semiconductores, además de tener estados brillante-excitón es posible encontrar estados que no interactúen con la luz, conocidos como excitones oscuros. Las reglas de selección de Pauli definen si un estado excitón es brillante u oscuro, es decir, brillante si el espín del electrón y el hueco son anti-paralelos, oscuro en el caso contrario. Los estados que no se acoplan a la luz, por lo tanto, están ópticamente inactivos y no sufren procesos de recombinación\footnote{Debido a que no pueden coexistir electrón y hueco en la banda de valencia con el mismo espín.} por medio de emisión de fotones, por lo tanto, tendrán un tiempo de vida más largo. Los estados oscuro-excitón son neutros, al igual que los estados brillante-excitón, y no sufren la influencia electrostática del entorno \parencite{Jimenez2017}. Estos estados son considerados para tener mejores propiedades al manejar la información cuántica \parencite{Poem2010}. Los estados oscuros y brillantes son degenerados, razón por la cuál no son accesibles directamente, por lo tanto, es posible acceder a ellos a través de un proceso indirecto mediado por $\vec{\mathbf{B}}$, donde $B_\parallel$ a la dirección de crecimiento del QD permite un cambio en la energía y $B_\perp$ a la dirección de crecimiento del QD se usa como un parámetro de control que mezcla los estados excitón con un rompimiento de simetría \parencite{Bayer2000}.

%Bin et al., 2020 
%\section{Manipulación de estados cuánticos}
Un tema principal de la ciencia moderna es la manipulación de estados cuánticos \parencite{Bin2020}. Se han sugerido sistemas para administrar la emisión de estados multi-fotón y multi-fonón. En el caso de los estados multi-fotón, se plantea la excitación desde el estado base hasta un estado $n$ a través de frecuencias elevadas. Posteriormente, se provoca una desexcitación estimulada hasta retornar al estado base, en un proceso con efecto cascada que puede liberar $n$ fotones. Estos fotones pueden estar correlacionados o no, dependiendo de ciertos parámetros preestablecidos.

Para los estados multi-fonón, una estrategia para la generación de múltiples fonones es mediante procesos de Stokes, o sea, empleando frecuencias de bandas laterales. Este procedimiento permite que, debido al acoplamiento del QD con la cavidad acústica, el electrón se desexcite y emita fonones, siempre que las frecuencias estén permitidas. Así, se pueden producir numerosos fonones con efecto cascada, los cuales pueden o no estar correlacionados.

%\subsubsection{Fonones}
Una línea de investigación en crecimiento popular es la física multifotónica \parencite{Kubanek2008, Ota2011} con aplicaciones potenciales: láseres multifotón \parencite{Gauthier1992}, superar el límite de difracción \parencite{DAngelo2001} y Metrología \parencite{Afek2010}, entre otras. Un esquema para generación directa de estados de $N$-fotones en el mismo modo (bundles de $N$-fotones) ha sido propuesta bajo la plataforma de la electrodinámica cuántica de cavidades, en inglés cavity Quantum Electrodynamics (cQED) \parencite{Muñoz2014}.

Así mismo, los fonones, análogos al fotón de la electrodinámica cuántica, pueden ser usados para almacenar, procesar y transducir información cuántica \parencite{Vargas2022}. La rapidez de las ondas acústicas\footnote{Es significativamente menor que la velocidad de la luz.} junto con la variedad de energías características de los fonones\footnote{Debido a que las cavidades fonónicas son bastante sintonizables con una gran variedad de rangos de frecuencia resonante, desde gigahertz (GHz) hasta terahertz (THz), ya han sido fabricadas \parencite{Borri2001} } que van desde megahertz (MHz) a terahertz (THz) \parencite{Wigger2021, Kettler2021}, en general, diferentes a las ópticas, los hace especialmente adecuados para comunicaciones a cortas distancias, como unos pocos cientos de micrómetros o menos, es decir, comunicación en chips \parencite{Gustafsson2014, Wan2021}. Por lo tanto, los fonones\footnote{También conocidos como cuantos de ondas mecánicas o de sonido}, surgen como fuertes candidatos para la ingeniería de dispositivos cuánticos de estado sólido y comunicaciones cuánticas en chip \parencite{Bin2020}. Además, estos modos vibracionales cuánticos de los sólidos tienen un gran potencial para ser usado en aplicaciones tecnológicas en metrología o procesamiento de la información cuántica \parencite{Zhang2018}.

Dentro de las ventajas que presentan los fonones están: son inmunes a las pérdidas de radiación dentro del campo electromagnético de vacío, debido a que necesitan un medio material para propagarse (en dispositivos de estado sólido, comúnmente) y muchas técnicas experimentales desarrolladas por físicos del estado sólido \parencite{Ask2019} ya están disponibles para las tareas de procesamiento de información cuántica con fonones.

Entre las aplicaciones de fonones THz se destacan la detección, debido a que tienen una longitud de onda comparable a las constantes de red, y obtención de imágenes a nanoescala como la detección de estructuras microscópicas subsuperficiales con precisión atómica \parencite{Bin2020}.

%avance en fonónica cuántica
Actualmente la fonónica cuántica ha progresado enormemente. Es posible implementar láseres fonónicos \parencite{Kabuss2012}, redes cuánticas fonónicas \parencite{Lemonde2018}, dispositivos cuánticos acústicos \parencite{Schütz2017}, hasta detectar la interacción electrón-fonón en puntos cuánticos dobles \parencite{Hartke2018}. Por lo tanto, se estudian los fonones y su interacción con otras excitaciones para probar la física fundamental en sistemas cuánticos de muchos cuerpos, y también en sistemas individuales \parencite{Lupke2022}.

% tarea importante
Una tarea importante de la fonónica, como un hito fundamental en el camino de los dispositivos cuánticos acústicos, es la generación de estados cuánticos multifonón. Por ejemplo, antibunching\footnote{Los estados cuánticos vienen desempaquetados, o desagrupados; en particular, quiere decir que cada fonón está definido por su propio estado y no hay un estado que los represente a todos como individuo} bundles\footnote{Los fononones, para este caso, se pueden empaquetar, o agrupar, sin perder sus propiedades cuánticas. Por lo tanto, la estadística que los representa es cuántica. Además, cada uno tiene el mismo modo.} y estados fonónicos N00N\footnote{$\ket{\psi_{N00N}}=(\ket{N}_a \ket{0}_b + e^{iN\theta} \ket{0}_a \ket{N}_b)/\sqrt{2}$ donde \(\ket{N}_a\) y \(\ket{N}_b\) son estados cuánticos (con número cuántico $N$) en los sistemas $a$ y $b$, respectivamente, y \(\ket{0}_a\) y \(\ket{0}_b\) son estados cuánticos en los sistemas $a$ y $b$ en su estado base (estado cero). El segundo término, \(e^{iN\theta} \ket{0}_a \ket{N}_b\), involucra un factor de fase \(e^{iN\theta}\). Esto indica que el sistema $a$ está en su estado base \(\ket{0}\) y el sistema $b$ está en el estado cuántico \(\ket{N}\), pero con un cambio de fase dado por \(e^{iN\theta}\). Por lo tanto, el estado $\ket{\psi_{N00N}}$ representa un estado cuántico en el que los sistemas $a$ y $b$ están correlacionados y pueden estar en una superposición de estados, dependiendo de los valores de \(N\) y \(\theta\).} podrían ser valiosos como fuentes de $N$-fonones \parencite{Chu2018} y para medidas de precisión cuántica acústica \parencite{Toyoda2015}, respectivamente. 

%Vargas et al., 2022
%\subsubsection{Estados multi-fonón}
Las fuentes altamente no clásicas son útiles para metrología cuántica \parencite{Zhang2018}, detección cuántica, tecnologías cuánticas tal como las memorias cuánticas y los transducers \parencite{Arrangoiz2018}. Para su realización es necesario estados multi-fonón, a pesar de que la mayoría de esfuerzos teóricos y experimentales se han enfocado en generar y controlar fonones individuales.

Un trabajo destacado con estados multi-fonón es el de \textcite{Bin2020} donde proponen un sistema físico, en acustodinámica cuántica, compuesto de un punto cuántico semiconductor, modelado como un sistema de dos niveles, acoplado al modo fonón de la cavidad acústica. El QD es bombeado coherentemente por un láser externo ajustando finamente su frecuencia para excitar oscilaciones gigante-Rabi, principalmente, entre el estado vacío y un estado excitón, acompañado por $N$ fonones en la cavidad acústica. Los autores muestran que los canales disipativos permiten la emisión de bundles de $N$-fonones, y luego analizan las estadísticas cuánticas de esta emisión, encontrando que, dependiendo de los parámetros Hamiltonianos y disipativos, pueden producir un láser de fonones o un cañón de fonones.

%por ejemplo lo que esta escrit en el articulo de Vladimier, computacion cuanrica, procesamiento de informacion, informacion cantica
\end{document}